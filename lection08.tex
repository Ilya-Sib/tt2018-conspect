\section{Лекция 8}
	\subsection{Ранг типа}
	\begin{definition}
		 Введем определение. Под {рангом типа} мы будем понимать число, получаемое по следующим правилам: \\
	    Rn(x) --- множество всех типов x\\
	    Rn(0) --- все типы без кванторов\\
	    Rn(x+1) = Rn(x) | Rn(x) $\rightarrow$ Rn(x+1) | $\forall\alpha.Rn(x+1)$
	\end{definition}
	 
	\textbf{Примеры}
	 1. $ \alpha\in $Rn(0) \\
	 2. $ \forall\alpha.\alpha \in$Rn(1)\\
	 3. $ (\forall\alpha.\alpha)\rightarrow(\forall\beta.\beta) \in$Rn(2), так как каждый тип вида $ \forall\alpha.\alpha \in$Rn(1), то по третьему правилу весь тип $ \in $Rn(2) \\

	\begin{definition}
		Тип с поверхностными кванторами --- это любой тип вида $ \forall\alpha.\tau $, где в $ \tau $ отсутствуют кванторы. Очевидно, что любой такой тип $ \in $Rn(1). Действительно, тип внутри квантора точно имеет ранг 0. Навешивание одного или нескольких кванторов всеобщности увеличит его ранг на единицу.
	\end{definition}
	 
	 \subsection{Типовая схема}
	 Возьмем только типы с поверхностными кванторами(из Rn(1)). \\
	 Также можно превратить любую формулу из Rn(1) в формулу с поверхностными кванторами. \\
	 Например:\\ 
	 $ \beta\rightarrow\forall\alpha.\alpha\equiv\forall\alpha.(\beta\rightarrow\alpha) $
	 \\
	 
	 \begin{definition}
	 	{Типовой схемой} назовем выражение вида: 

\begin{center}
		 $ \sigma\equiv\forall\alpha_1.\forall\alpha_2.....\forall\alpha_n.t $ где t$ \in $Rn(0)
\end{center}
	 	 \end{definition}
 	 
	 Также будем считать, что $ \sigma_1 <= \sigma_2 $\\
	 ($\sigma_1$ является спецификацией $\sigma_2$) если: \\\\
	 $\sigma_2\equiv\forall\alpha_1...\forall\alpha.\tau_1$\\
	 $\sigma_1\equiv\forall\beta_1...\beta_n.\tau_1[\alpha_1:=\Theta_1]...[\alpha_n:=\Theta_n]  $
	 
	 \noindent Например:\\
	 $ \forall\beta_1.\forall\beta_2.(\beta_1\rightarrow\beta_2)\rightarrow(\beta_1\rightarrow\beta_2) $\\
	 является спецификацией $ \forall\alpha.\alpha\rightarrow\alpha $\\
	 \subsection{Экзистенциальные типы}
	 1) $\dfrac{\Gamma\vdash\phi[\alpha:=\theta]}{\Gamma\vdash\exists\alpha.\phi}$\\ \\
	 2) $\dfrac{\Gamma\vdash\exists\alpha.\phi\qquad\Gamma,\phi\vdash\psi}{\Gamma\vdash\psi}$
	 
	  Экзистенциальные типы это способ инкапсуляции данных. Предположим, что у нас есть стек с хранилищем типа $\alpha$, у которого определены следующие операции:\\\\
	 \textbf{empty}: $\alpha$\\
	 \textbf{push}: $\alpha\&\nu\rightarrow\alpha$\\
	 \textbf{pop}: $\alpha\rightarrow\alpha\&\nu$\\
	 
	 Тогда очевидно, что тип stack$\equiv\alpha\&(\alpha\&\nu\rightarrow\alpha)\&(\alpha\rightarrow\alpha\&\nu)$.
	 Но что если мы реализовали хранилище как-то по-особенному, не меняя типов операций. Мы хотим скрыть данные о реализации, в частности о типе $\alpha$. Вместо деталей просто скажем, что существует интерфейс, удовлетворяющий такому типу:\\$\exists\alpha.\alpha\&(\alpha\&\nu\rightarrow\alpha)\&(\alpha\rightarrow\alpha\&\nu)$
	 
	 \subsection{Абстрактные типы}	 
	 Предположим, что мы захотим создать стек, в котором лежат целые числа. Рассмотрим, как тогда будет выглядеть тип созданного стека: \\
	 \textbf{stack}$\equiv\forall\nu.\exists\alpha.\alpha\&(\alpha\&\nu\rightarrow\alpha)\&(\alpha\rightarrow\alpha\&\nu)$\\
 	По аналогии с правилом удаления квантора существования, можно определить правила вывода для выражений абстрактных типов: \\

	
 	$\dfrac{\Gamma \vdash M : \varphi[\alpha := \theta]}{\Gamma\vdash (\text{pack } M, \theta \text{ to } \exists \alpha . \varphi) : \exists \alpha.\varphi}\\ \\ \\
	$ Это правило вывода позволяет скрыть реализацию стека, так как если $\alpha$ --- это тип стека, то $\alpha[\nu := \theta]$ --- его конкретная реализация, например ArrayStack, LinkedListStack и подобные \\ \\
 	 $
 	\dfrac{\Gamma \vdash M : \exists \alpha . \varphi\qquad\Gamma, x : \varphi \vdash N : \psi}{\Gamma \vdash \text{abstype } \alpha \text{ with } x:\varphi \text{ in } M \text{ is } N:\psi}
	(\alpha \notin FV(\Gamma, \psi))$
	\\ \\
	Это правило вывода соответствует виртуальному вызову стека какой-то реализации, например: 
	\begin{verbatim}
		foo(Stack s) {
			...
		}
	\end{verbatim}
	Поскольку выводимые формулы выглядят слишком громоздко, перепишем их, вспомнив, что: \\
	$\exists\alpha.\sigma\equiv\forall\beta.(\forall\alpha.\sigma\rightarrow\beta)\rightarrow\beta$\\\\
	Тогда: \\
	$	\text{\textbf{pack} } M, \theta \text{ \textbf{to} } \exists \alpha . \varphi =
		\Lambda \beta . \lambda x^{\forall \alpha . \varphi \to \beta} . x \theta M \\
		\text{\textbf{abstype} } \alpha \text{ \textbf{with} } x:\varphi \text{ \textbf{in} } M \text{ \textbf{is} } N:\psi =
		M \psi (\Lambda \alpha . \lambda x ^ \varphi . N)
	$
	
	 \subsection{Типовая система Хиндли-Милнера}
	   
	 Начнем с определения типа. \underline{Тип в системе Хиндли-Милнера}: \\ \\
	 Монотип --- выражение в грамматике вида $\tau::=\alpha|\tau\rightarrow\tau|(\tau)$\\
	 Политип --- выражение в грамматике вида $\sigma::=\tau|\forall\alpha.\sigma$\\
	 
	 \noindent Поэтому типы вида $\alpha\rightarrow\forall\beta.\beta$ - некорректны в системе ХМ\\
	 
	 \noindent Грамматика в системе Хиндли-Милнера имеет вид:\\
	 
	 \begin{center}
	 $\Lambda::=x | \lambda x.\Lambda | \Lambda\Lambda | (\Lambda) | \text{let x = }\Lambda \text{ in }\Lambda $ 
	 \end{center}
 		
 	 \noindent Обозначим контекст $\Gamma$ без типа x как $\Gamma_x$\\
 	 В новой системе получаем следующие правила вывода: \\ \\
 	 
 	 \noindent 1. Тавтология $\dfrac{}{\Gamma\vdash x.\phi}$ \\\\
 	
 	 \noindent 2. Уточнение $\dfrac{\Gamma\vdash e:\sigma\qquad\sigma' <= \sigma}{\Gamma\vdash e:\sigma'}$ \\\\
 	 
 	 \noindent 3. Обобщение $\dfrac{\Gamma\vdash e:\sigma}{\Gamma\vdash e:\forall\alpha.\sigma}$ \\ \\
 	 \noindent 4. Абстракция $\dfrac{\Gamma_x, x:\tau'\vdash e:\tau}{\Gamma\vdash \lambda x.e:\tau'\rightarrow\tau}$ \\ \\
 	 \noindent 5. Применение $\dfrac{\Gamma\vdash e:\tau'\rightarrow\tau\qquad\Gamma\vdash e':\tau'}{\Gamma\vdash e\ e':\tau}$ \\ \\
 	 \noindent 6. Let $\dfrac{\Gamma\vdash e:\sigma\qquad\Gamma_x,x:\sigma\vdash e':\tau}{\Gamma\vdash\text{ \textbf{let} } x = e \textbf{ in }e':\tau}$ \\
 	 
 	 Хотя в системе Хиндли-Милнера (как и во всех рассматриваемых нами типовых системах) нельзя типизировать $\mathcal{Y} $-комбинатор,
 	 можно добавить его, расширив язык.
 	 Давайте определим его как $\mathcal{Y} f = f \left(\mathcal{Y} f\right)$.
 	 Какой у него должен быть тип? Пусть $\mathcal{Y}$ принимает $f$ типа $\alpha$, и возвращает нечто типа $\beta$,
 	 то есть $\mathcal{Y}: \alpha\to\beta$.
 	 Функция $f$ должна принимать то же, что возвращает $\mathcal{Y}$, так как результат $\mathcal{Y}$ передаётся в $f$,
 	 и возвращать она должна то же, что возвращает $\mathcal{Y}$, так как тип выражений с обеих сторон равенства должен быть одинаковый,
 	 то есть $f : \beta\to\beta$
 	 Кроме того, $\alpha$ и тип $f$ это одно и то же, $\alpha=\beta\to\beta$.
 	 После подстановки и заключения свободной переменной под квантор получаем $\mathcal{Y} : \forall\beta.(\beta\to\beta)\to\beta$.
 	 
 	 Через такой $\mathcal{Y}$ можно определять рекурсивные функции, и они будут типизироваться.
 	 
 
 	 

