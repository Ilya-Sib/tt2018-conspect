		\section{Лекция 6 \\ Реконструкция типов в просто типизированном \\ $\lambda$-исчислении, комбинаторы}
		\subsection{Алгоритм вывода типов}
					
			Пусть есть: $?\:\vdash \:A\: :\: ?$, хотим найти пару $\big \langle \text{контекст}, \text{тип} \big \rangle$\par
	\textbf{Алгоритм:}
	\begin{enumerate}
		\item Рекурсия по структуре формулы\par Построить по формуле $A$ пару $\big \langle E, \tau\big \rangle$, где\par $E-$система уравнений, $\tau-$тип $A$
		\item Решение уравнения, получение подстановки $S$ и из решения $E$ и $S\:(\tau)$ получение ответа	
	\end{enumerate}
		Т.е. необходимо свести вывод типа к алгоритму унификации.\par
		\begin{oun_paragraph}Рассмотрим 3 случая\end{oun_paragraph}
			\textbf{Обозначение } $\rightarrow-$ алгебраический тип 
			\begin{enumerate}
				\item $A\equiv x\implies\:\big \langle \{\}, \alpha_A\big\rangle$, где $\{\}-$пустой контекст, $\alpha_A-$новая переменная, нигде не встречавшаяся до этого в формуле
				\item $A\equiv P\:Q\implies\big \langle E_P\cup E_Q\cup \{\tau_P=\:\rightarrow\:(\tau_Q\:\alpha_A)\}, \alpha_A\big \rangle$, где $\alpha_A-$новая переменная
				\item $A\equiv\lambda x.P\implies\big\langle E_P,\alpha_x\:\rightarrow\:\tau_P\big\rangle$
			\end{enumerate}
		\begin{oun_paragraph}Алгоритм унификации\end{oun_paragraph} 
			Рассмотрим $E-$систему уравнений, запишем все уравнения в алгебраическом виде, т.е. \par $\alpha\:\rightarrow\:\beta\:\Leftrightarrow\:\rightarrow\:\alpha\:\beta$, затем применяем алгоритм унификации.
	\begin{lemma}
	Рассмотрим терм $M$ и пару $\big\langle E_M, \tau_M\big\rangle$, Если $\Gamma\:\vdash M:\rho$, то существует:
	\end{lemma}	
		\begin{enumerate}
		\item $S-$решение $E_M$ тогда $\Gamma=\{x : S(\alpha_x)\:|\:x\in FV(M)\}$, $FV-$множество свободных переменных в терме $M$, $\alpha_x-$ переменная, полученная при разборе терма $M$\par
		$\rho\:=\:S(\tau_M)$
		\item Если $S-$ решение $E_M$, то $\Gamma\:\vdash M:\rho$,
		\begin{proof}индукция по структуре терма $M$

			\begin{enumerate}
				\item Если $M\equiv x$, то так как решение существует, то существует и $S(\alpha_x)$, что: \par$\Gamma,\:x:S(\alpha_x)\:\vdash \:x:S(\alpha_x)$
				\item Если $M\equiv \lambda x.\:P$, то по индукции уже известен тип $P$, контекст $\Gamma$ и тип $x$, тогда: $${\Gamma,\:x:\:S(\alpha_x)\:\vdash \:P:\:S(\alpha_P) \over \Gamma\:\vdash \:\lambda x.\:P:S(\alpha_x)\:\rightarrow\:S(\alpha_P)}$$
				\item Если $M\equiv P\:Q$, то по индукции: $${ \Gamma\:\vdash \:P:\:S(\alpha_P)\equiv\tau_1\:\rightarrow\:\tau_2 \hspace{4em} \Gamma\:\vdash \:Q:S(\alpha_Q)\equiv\tau_1\over \Gamma\:\vdash \:P\:Q:\tau_2}$$
			\end{enumerate}					
		\end{proof}
	\end{enumerate}				
		 $\big \langle\Gamma,\tau\big\rangle\:-\:$основная пара для терма $M$, если 
		 \begin{enumerate}
			\item $\Gamma\:\vdash M:\tau$
			\item Если $\Gamma'\:\vdash M:\tau'$, то существует $S:\:S(\Gamma)\:\subset\:\Gamma'$
		 \end{enumerate}
		 \begin{example}
		\end{example}
		 	Рассмотрим терм: $\lambda f\:\lambda x.\:f(f(x))$, построим и пронумеруем его дерево разбора:\par
\begin{tikzpicture}
	\node [fill=none, draw=none] (A) at (4,0) {$\lambda f\:\lambda x.\:f(f(x))\enspace{\color{blue} (7)}$};
	\node [fill=none, draw=none] (B) at (7.5,-1) {$\lambda x.\:f(f(x))\enspace{\color{blue} (6)}$};
	\node [fill=none, draw=none] (C) at (2,-1) {$\lambda f$};
	\node [fill=none, draw=none] (D) at (10.5,-2) {$f(f(x))\enspace{\color{blue} (5)}$};
	\node [fill=none, draw=none] (E) at (5.5,-2) {$\lambda x$};
	\node [fill=none, draw=none] (F) at (13.5,-3) {$f(x)\enspace{\color{blue} (4)}$};
	\node [fill=none, draw=none] (G) at (8.5,-3) {$f\enspace{\color{blue} (3)}$};
	\node [fill=none, draw=none] (H) at (11.5,-4) {$f\enspace{\color{blue} (2)}$};
	\node [fill=none, draw=none] (I) at (16.5,-4) {$x\enspace{\color{blue} (1)}$};
	\path [->] (A) edge node[left] {} (B);
	\path [->] (A) edge node[left] {} (C);	
	\path [->] (B) edge node[left] {} (D);
	\path [->] (B) edge node[left] {} (E);
	\path [->] (D) edge node[left] {} (F);
	\path [->] (D) edge node[left] {} (G);
	\path [->] (F) edge node[left] {} (H);	
	\path [->] (F) edge node[left] {} (I);
\end{tikzpicture}\par


\begin{enumerate}
	\item $\big\langle E_1, \tau_1\big\rangle=\big \langle \{\},\:\alpha_x \big \rangle$
	\item $\big\langle E_2, \tau_2\big\rangle=\big \langle \{\},\:\alpha_f \big \rangle$
	\item $\big\langle E_3, \tau_3\big\rangle=\big \langle \{\},\:\alpha_f \big \rangle$
	\item $\big\langle E_4, \tau_4\big\rangle=\big \langle \{\alpha_f=\:\rightarrow(\alpha_x\:\alpha_1)\}, \alpha_1 \big \rangle$
	\item $\big\langle E_5, \tau_5\big\rangle=\big \langle \begin{Bmatrix}
	\alpha_f=\:\rightarrow(\alpha_x\:\alpha_1)\\
	\alpha_f=\:\rightarrow(\alpha_1\:\alpha_2)
\end{Bmatrix},\:\alpha_2 \big \rangle$
	\item $\big\langle E_6, \tau_6\big\rangle=\big \langle \begin{Bmatrix}\alpha_f=\:\rightarrow(\alpha_x\:\alpha_1)\\ \alpha_f=\:\rightarrow(\alpha_1\:\alpha_2)\end{Bmatrix},\:\alpha_x\rightarrow\alpha_2 \big \rangle$
	\item $\big\langle E_7, \tau_7\big\rangle=\big \langle \begin{Bmatrix}\alpha_f=\:\rightarrow(\alpha_x\alpha_1)\\ \alpha_f=\:\rightarrow(\alpha_1\alpha_2)\end{Bmatrix},\:\alpha_f\rightarrow(\alpha_x\rightarrow\alpha_2) \big \rangle$
\end{enumerate}
	$E=\begin{Bmatrix}\alpha_f=\:\rightarrow\:(\alpha_x\:\alpha_1)\\ \alpha_f=\:\rightarrow\:(\alpha_1\:\alpha_2)\end{Bmatrix}$, решим полученную систему:\par 
	\begin{enumerate}
		\item Решим систему:\par 
			\begin{enumerate}
			\item \[\begin{cases}
				\alpha_f=\rightarrow(\alpha_x\:\alpha_1)&\\
				\alpha_f=\rightarrow(\alpha_1\:\alpha_2)&\\
			\end{cases}\]
			\item \[
				\begin{cases}
				\rightarrow(\alpha_1\:\alpha_2)=\rightarrow(\alpha_x\:\alpha_1)
				\end{cases}\]
			\item \[
				\begin{cases}
				\alpha_1=\alpha_x&\\
				\alpha_2=\alpha_1
				\end{cases}\]
			\item \[
				\begin{cases}
				\alpha_1=\alpha_x&\\
				\alpha_2=\alpha_x
				\end{cases}\]
\end{enumerate}					
		\item  Получим \[S=\begin{cases}
						\alpha_f=\rightarrow(\alpha_x\:\alpha_1)&\\
						\alpha_1=\alpha_x&\\
						\alpha_2=\alpha_x&\\
				\end{cases}\]
		\item $\Gamma=\{\}$, так как в заданной формуле нет свободных переменных
		\item Тип  терма $\lambda\:f\lambda\:x.f(f(x))$ является результатом подстановки\par $S(\rightarrow\:\alpha_f\:(\alpha_x\rightarrow\alpha_2))$, получаем $\tau=(\alpha_x\rightarrow\alpha_x)\rightarrow(\alpha_x\rightarrow\alpha_x)$
	\end{enumerate}
	\subsection{Сильная и слабая нормализации}
	\begin{definition}Если существует последовательность редукций, приводящая терм $M$ в нормальную форму, то $M-$слабо нормализуем. (Т.е. при редуцировании терма $M$ мы можем не прийти в н.ф.)\end{definition} 
	\begin{definition}Если не существует бесконечной последовательности редукций терма $M$, то терм $M-$ сильно нормализуем.\end{definition} 
	 \begin{statement}
	 \end{statement}
	 \begin{enumerate}
	 	\item $KI\Omega-$ слабо нормализуема
	 		\begin{example}\end{example}
	 		Перепишем $KI\Omega$ как ${\color{red}(}(\lambda x\:\lambda y.\:x)(\lambda x.\:x){\color{red})}{\color{blue}(}((\lambda x.\:x\:x)(\lambda x.\:x\:x)){\color{blue})}$, очевидно, что этот терм можно средуцировать двумя разными способами:\\
	 		\begin{enumerate}
	 			\item Сначала редуцируем красную скобку 
	 				\begin{enumerate}
	 					\item ${\color{red}(}(\lambda x\:\lambda y.\:x)(\lambda x.\:x){\color{red})}{\color{blue}(}((\lambda x.\:x\:x)(\lambda x.\:x\:x)){\color{blue})}$
	 					\item ${\color{red}(}(\lambda y.\:(\lambda x.\:x)){\color{red})}{\color{blue}(}((\lambda x.\:x\:x)(\lambda x.\:x\:x)){\color{blue})}$
	 					\item $(\lambda x.\:x)$
	 				\end{enumerate}
	 				Видно, что в этом случае количество шагов конечно.
	 			\item Редуцируем синюю скобку. Очевидно, что комбинатор $\Omega$ не имеет нормальной формы, тогда понятно, что в этом случае терм $KI\Omega$ никогда не средуцируется в нормальную форму.
	 		\end{enumerate}
	 	\item $\Omega-$ не нормализуема
	 	\item $II-$ сильно нормализуема
	 \end{enumerate}
	 \begin{lemma}Сильная нормализация влечет слабую.
	 \end{lemma}
	 \subsection{Выразимость комбинаторов}
	 \begin{statement}Для любого $\lambda$-выражения без свободных переменных существует $\beta$-эквивалентное ему выражение, записываемое только с помощью комбинаторов $S$ и $K$, где\end{statement}
	 $S\:=\:\lambda x\:\lambda y\:\lambda z.\:(x\:z)\:(y\:z)\: : \:(a\:\rightarrow\:b\:\rightarrow\:c)\:\rightarrow\:(a\:\rightarrow\:b)\:\rightarrow\:a\:\rightarrow\:c$\par 
	 $K=\lambda x\:\lambda y.\:x\: : a\:\rightarrow\:b\:\rightarrow\:a$\par 
	 \begin{statement}Комбинаторы $S$ и $K$ являются аксиомами в ИИВ\end{statement}
	\begin{statement}Соотношение комбинаторов с $\lambda$ исчислением:\end{statement}
	\begin{enumerate}
		\item $T(x)=x$
		\item $T(P\:Q)=T(P)\:T(Q)$
		\item $T(\lambda\:x.P)=K(T(P)),\enspace x\not\in FV(P)$
		\item $T(\lambda\:x.x)=I$
		\item $T(\lambda\:x\lambda\:y.P)=T(\lambda\:x.\:T(\lambda\:y.P))$
		\item $T(\lambda\:x.P\:Q)=S\:\:T(\lambda\:x.P)\:T(\lambda\:x.Q)$
	\end{enumerate}
	\begin{statement}Альтернативный базис:\end{statement}
	 \begin{enumerate}
		\item $B=\lambda x\:\lambda y\:\lambda z.\:x\:(y\:z)\: : \:(a\:\rightarrow\:b)\:\rightarrow\:(c\:\rightarrow\:a)\:\rightarrow\:c\:\rightarrow\:b$
		\item $C=\lambda x\:\lambda y\:\lambda z.\:((x\:z)\:y)\: : \:(a\:\rightarrow\:b\:\rightarrow\:c)\:\rightarrow\:b\:\rightarrow\:a\:\rightarrow\:c$
		\item $K=\lambda x\:\lambda y.\:x : a\:\rightarrow\:b\:\rightarrow\:a$
		\item $W=\lambda x\:\lambda y.\:((x\:y)\:y)\: : \: (a\:\rightarrow\:a\:\rightarrow\:b)\:\rightarrow\:a\:\rightarrow\:b$
	 \end{enumerate}
